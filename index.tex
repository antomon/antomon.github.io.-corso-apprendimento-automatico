% Options for packages loaded elsewhere
\PassOptionsToPackage{unicode}{hyperref}
\PassOptionsToPackage{hyphens}{url}
\PassOptionsToPackage{dvipsnames,svgnames,x11names}{xcolor}
%
\documentclass[
  letterpaper,
  DIV=11,
  numbers=noendperiod]{scrreprt}

\usepackage{amsmath,amssymb}
\usepackage{iftex}
\ifPDFTeX
  \usepackage[T1]{fontenc}
  \usepackage[utf8]{inputenc}
  \usepackage{textcomp} % provide euro and other symbols
\else % if luatex or xetex
  \usepackage{unicode-math}
  \defaultfontfeatures{Scale=MatchLowercase}
  \defaultfontfeatures[\rmfamily]{Ligatures=TeX,Scale=1}
\fi
\usepackage{lmodern}
\ifPDFTeX\else  
    % xetex/luatex font selection
\fi
% Use upquote if available, for straight quotes in verbatim environments
\IfFileExists{upquote.sty}{\usepackage{upquote}}{}
\IfFileExists{microtype.sty}{% use microtype if available
  \usepackage[]{microtype}
  \UseMicrotypeSet[protrusion]{basicmath} % disable protrusion for tt fonts
}{}
\makeatletter
\@ifundefined{KOMAClassName}{% if non-KOMA class
  \IfFileExists{parskip.sty}{%
    \usepackage{parskip}
  }{% else
    \setlength{\parindent}{0pt}
    \setlength{\parskip}{6pt plus 2pt minus 1pt}}
}{% if KOMA class
  \KOMAoptions{parskip=half}}
\makeatother
\usepackage{xcolor}
\setlength{\emergencystretch}{3em} % prevent overfull lines
\setcounter{secnumdepth}{5}
% Make \paragraph and \subparagraph free-standing
\ifx\paragraph\undefined\else
  \let\oldparagraph\paragraph
  \renewcommand{\paragraph}[1]{\oldparagraph{#1}\mbox{}}
\fi
\ifx\subparagraph\undefined\else
  \let\oldsubparagraph\subparagraph
  \renewcommand{\subparagraph}[1]{\oldsubparagraph{#1}\mbox{}}
\fi


\providecommand{\tightlist}{%
  \setlength{\itemsep}{0pt}\setlength{\parskip}{0pt}}\usepackage{longtable,booktabs,array}
\usepackage{calc} % for calculating minipage widths
% Correct order of tables after \paragraph or \subparagraph
\usepackage{etoolbox}
\makeatletter
\patchcmd\longtable{\par}{\if@noskipsec\mbox{}\fi\par}{}{}
\makeatother
% Allow footnotes in longtable head/foot
\IfFileExists{footnotehyper.sty}{\usepackage{footnotehyper}}{\usepackage{footnote}}
\makesavenoteenv{longtable}
\usepackage{graphicx}
\makeatletter
\def\maxwidth{\ifdim\Gin@nat@width>\linewidth\linewidth\else\Gin@nat@width\fi}
\def\maxheight{\ifdim\Gin@nat@height>\textheight\textheight\else\Gin@nat@height\fi}
\makeatother
% Scale images if necessary, so that they will not overflow the page
% margins by default, and it is still possible to overwrite the defaults
% using explicit options in \includegraphics[width, height, ...]{}
\setkeys{Gin}{width=\maxwidth,height=\maxheight,keepaspectratio}
% Set default figure placement to htbp
\makeatletter
\def\fps@figure{htbp}
\makeatother
% definitions for citeproc citations
\NewDocumentCommand\citeproctext{}{}
\NewDocumentCommand\citeproc{mm}{%
  \begingroup\def\citeproctext{#2}\cite{#1}\endgroup}
\makeatletter
 % allow citations to break across lines
 \let\@cite@ofmt\@firstofone
 % avoid brackets around text for \cite:
 \def\@biblabel#1{}
 \def\@cite#1#2{{#1\if@tempswa , #2\fi}}
\makeatother
\newlength{\cslhangindent}
\setlength{\cslhangindent}{1.5em}
\newlength{\csllabelwidth}
\setlength{\csllabelwidth}{3em}
\newenvironment{CSLReferences}[2] % #1 hanging-indent, #2 entry-spacing
 {\begin{list}{}{%
  \setlength{\itemindent}{0pt}
  \setlength{\leftmargin}{0pt}
  \setlength{\parsep}{0pt}
  % turn on hanging indent if param 1 is 1
  \ifodd #1
   \setlength{\leftmargin}{\cslhangindent}
   \setlength{\itemindent}{-1\cslhangindent}
  \fi
  % set entry spacing
  \setlength{\itemsep}{#2\baselineskip}}}
 {\end{list}}
\usepackage{calc}
\newcommand{\CSLBlock}[1]{\hfill\break\parbox[t]{\linewidth}{\strut\ignorespaces#1\strut}}
\newcommand{\CSLLeftMargin}[1]{\parbox[t]{\csllabelwidth}{\strut#1\strut}}
\newcommand{\CSLRightInline}[1]{\parbox[t]{\linewidth - \csllabelwidth}{\strut#1\strut}}
\newcommand{\CSLIndent}[1]{\hspace{\cslhangindent}#1}

\KOMAoption{captions}{tableheading}
\makeatletter
\@ifpackageloaded{bookmark}{}{\usepackage{bookmark}}
\makeatother
\makeatletter
\@ifpackageloaded{caption}{}{\usepackage{caption}}
\AtBeginDocument{%
\ifdefined\contentsname
  \renewcommand*\contentsname{Table of contents}
\else
  \newcommand\contentsname{Table of contents}
\fi
\ifdefined\listfigurename
  \renewcommand*\listfigurename{List of Figures}
\else
  \newcommand\listfigurename{List of Figures}
\fi
\ifdefined\listtablename
  \renewcommand*\listtablename{List of Tables}
\else
  \newcommand\listtablename{List of Tables}
\fi
\ifdefined\figurename
  \renewcommand*\figurename{Figure}
\else
  \newcommand\figurename{Figure}
\fi
\ifdefined\tablename
  \renewcommand*\tablename{Table}
\else
  \newcommand\tablename{Table}
\fi
}
\@ifpackageloaded{float}{}{\usepackage{float}}
\floatstyle{ruled}
\@ifundefined{c@chapter}{\newfloat{codelisting}{h}{lop}}{\newfloat{codelisting}{h}{lop}[chapter]}
\floatname{codelisting}{Listing}
\newcommand*\listoflistings{\listof{codelisting}{List of Listings}}
\makeatother
\makeatletter
\makeatother
\makeatletter
\@ifpackageloaded{caption}{}{\usepackage{caption}}
\@ifpackageloaded{subcaption}{}{\usepackage{subcaption}}
\makeatother
\ifLuaTeX
  \usepackage{selnolig}  % disable illegal ligatures
\fi
\usepackage{bookmark}

\IfFileExists{xurl.sty}{\usepackage{xurl}}{} % add URL line breaks if available
\urlstyle{same} % disable monospaced font for URLs
\hypersetup{
  pdftitle={Corso apprendimento automatico},
  pdfauthor={Antonio Montano},
  colorlinks=true,
  linkcolor={blue},
  filecolor={Maroon},
  citecolor={Blue},
  urlcolor={Blue},
  pdfcreator={LaTeX via pandoc}}

\title{Corso apprendimento automatico}
\author{Antonio Montano}
\date{2024-03-30}

\begin{document}
\maketitle

\renewcommand*\contentsname{Table of contents}
{
\hypersetup{linkcolor=}
\setcounter{tocdepth}{2}
\tableofcontents
}
\bookmarksetup{startatroot}

\chapter*{Prefazione}\label{prefazione}
\addcontentsline{toc}{chapter}{Prefazione}

\markboth{Prefazione}{Prefazione}

\part{Introduzione}

Un'introduzione

\hfill\break

\part{1}

\chapter{Apprendimento automatico}\label{apprendimento-automatico}

Un'introduzione

\hfill\break

\chapter{1}\label{section-1}

\part{Basi dell'apprendimento supervisionato}

Un'introduzione

\hfill\break

\chapter{Percettrone}\label{percettrone}

Un'introduzione

\hfill\break

\section{Obiettivo della lezione}\label{obiettivo-della-lezione}

\begin{enumerate}
\def\labelenumi{\arabic{enumi}.}
\tightlist
\item
  Studiamo un algoritmo che è una pietra miliare dell'apprendimento
  delle macchine e che presenta concetti come l'addestramento
  supervisionato, l'ingegneria delle caratteristiche e la
  generalizzazione, che sono del tutto generali.
\item
  È semplice abbastanza perché si possa essere introdotti in uno dei
  problemi più importanti dell'apprendimento delle macchine, cioè quello
  della classificazione lineare, nel suo caso più semplice, quella
  binaria.
\end{enumerate}

\section{Prerequisiti}\label{prerequisiti}

\begin{enumerate}
\def\labelenumi{\arabic{enumi}.}
\tightlist
\item
  Conoscenza basilari di geometria, algebra e teoria delle funzioni di
  più variabili.
\item
  Capacità di comprensione delle componenti principali di un algoritmo.
\end{enumerate}

\section{Introduzione}\label{introduzione-1}

Il \textbf{\texttt{percettrone}}, introdotto pubblicamente per la prima
volta da Frank Rosenblatt nel 1958, ha giocato un ruolo cruciale nello
sviluppo dell'apprendimento delle macchine. Nonostante sia uno dei
modelli più semplici di apprendimento supervisionato, la sua
introduzione ha segnato l'inizio di un'epoca di grande interesse e
sviluppo nel campo delle reti neurali e dell'apprendimento automatico.\\
Il percettrone è, in breve, un algoritmo per
l'\texttt{apprendimento\ supervisionato\ di\ classificazioni\ binarie}.
In altre parole, riceve in ingresso un vettore di caratteristiche di un
modello fisico e produce un singolo risultato binario. Questo risultato
è determinato dalla somma pesata degli ingressi, che passa attraverso
una funzione opportuna, detta di attivazione per riferimento al neurone
di McCulloch-Pitts, che ha la forma a gradino per restituire un
risultato binario.\\
La forza del percettrone risiede nella sua capacità di apprendere i pesi
ottimali dai dati di addestramento, cioè dati di cui è noto
perfettamente sia il vettore di ingresso che il risultato (cioè la
classe binaria, dato che tali risultati possono essere solo due).
L'algoritmo di apprendimento, per ogni campione di addestramento,
predice la classe di risultato. Se la predizione è corretta, i pesi non
vengono modificati. Se la predizione è errata, i pesi vengono aggiornati
aggiungendo o sottraendo il vettore di ingresso, a seconda che la
predizione sia stata troppo bassa o troppo alta.\\
Rosenblatt ha dimostrato che se i dati di addestramento sono linearmente
separabili, allora l'algoritmo di addestramento convergerà a una
soluzione che classifica correttamente tutti i campioni di
addestramento, in un tempo finito. Block e Novikov hanno anche
determinato un limite superiore a tale tempo. Nonostante le sue
limitazioni, come l'incapacità di gestire dati che non sono linearmente
separabili, l'importanza del percettrone non deve essere sottovalutata.
Ha introdotto l'idea fondamentale che le macchine possono apprendere da
dati, e ha gettato le basi per lo sviluppo di modelli di apprendimento
automatico più sofisticati, tra cui le reti neurali multistrato e le
macchine a vettori di supporto.\\
Infatti, prima del percettrone, i modelli di neuroni artificiali, come
il modello di McCulloch-Pitts, erano statici, nel senso che i loro pesi
(o le loro connessioni sinaptiche) erano fissi e non cambiavano nel
tempo. Questi modelli potevano eseguire calcoli, ma non erano in grado
di adattarsi o apprendere dai dati, cioè la conoscenza era `predefinita'
nella rete neurale. Rosenblatt ha introdotto l'idea che i pesi delle
connessioni sinaptiche possano essere modificati in base all'esperienza,
in modo simile a come si pensava che i neuroni biologici si adattassero
e apprendessero nel cervello. Questo è stato un passo fondamentale verso
la creazione di modelli di apprendimento delle macchine che potessero
imparare dai dati e migliorare così le loro prestazioni nel tempo.\\
Rosenblatt è probabile che abbia chiamato questo algoritmo,
\texttt{perceptron} per sottolineare la sua capacità di `percepire' la
struttura nei dati di ingresso. Infatti, la parola perceptron richiama
`perception', percezione, che è il processo cognitivo utilizzato per
interpretare le informazioni sensoriali.

The Shallow and the Deep --\textgreater{} LIBRO

\section{Descrizione dell'algoritmo}\label{descrizione-dellalgoritmo}

Innanzitutto, distinguiamo tre fasi di esecuzione dell'algoritmo: 1.
\texttt{Fase\ di\ addestramento}: usiamo un certo numero di coppie di
ingressi e corrispondente uscita, che chiameremo, rispettivamente,
caratteristiche e classe, per calcolare dei parametri dell'algoritmo, i
pesi, fino a che una certa condizione di qualità globale dell'algoritmo
non sia soddisfatta. 2. \texttt{Fase\ di\ validazione}: adoperando delle
coppie non utilizzate in addestramento, si valuta la qualità della
predizione senza modificare i parametri del'algoritmo. 3.
\texttt{Fase\ di\ generalizzazione}: diamo in ingresso all'algoritmo
delle caratteristiche di cui non conosciamo a priori la classe `vera' e
ne otteniamo una `predetta'.

In generale, le caratteristiche presenti in ogni ingresso sono in un
numero prefissato positivo, che può essere anche arbitrariamente grande,
mentre l'uscita sarà sempre un valore unico scelto tra due possibilità.
Se rappresentiamo ogni ingresso con \(\mathbf{x}\), la corrispondente
classe `vera' con \(y\), i pesi con \(\mathbf{w}\) e, infine, la classe
predetta con \(\bar{y}\), allora nel diagramma seguente (Figura 1) è
riassunta la fase di addestramento del percettrone. \{\{
insert\_image(target, `1', image\_location,
`Percettrone-Diagramma-addestramento.png', `Figura 1: Diagramma del
percettrone in fase di addestramento', `Diagramma del percettrone in
fase di addestramento') \}\}

In ogni esecuzione di addestramento, forniremo in ingresso un gruppo di
caratteristiche distinte prese dalla totalità degli ingressi
disponibili, di cui conosciamo anche le rispettive classi. Tutte le
coppie di caratteristiche e relative classi compongono
l'\textbf{\texttt{insieme\ di\ addestramento}} e ogni suo elemento si
chiama \textbf{\texttt{campione}}. Nel diagramma le caratteristiche
entrano in un nodo dove viene calcolato il valore \(z\), ottenuto come
somma pesata delle caratteristiche e usando dei pesi ricavati
dall'esecuzione appena precedente. \(z\) è, a sua volta, l'ingresso di
un secondo nodo con la funzione \(\mathcal{H}\), detta
\texttt{funzione\ di\ attivazione}, che ne assegna la classe
corrispondente.\\
Nel caso del percettrone, la funzione di attivazione è tale che per
valori non negativi la classe è \(+1\), altrimenti, per quelli negativi,
è \(-1\). La funzione \(\mathcal{H}\) restituisce sempre valori
numerici, ma, generalmente, ad ognuno è associato, in esclusiva, un
oggetto del modello fisico che ha un nome distintivo, definito come
\texttt{etichetta}. Classe ed etichetta sono usati in modo
intercambiabile, data la corrispondenza 1 ad 1. Il valore predetto
\(\bar{y}\) viene confrontato in un terzo nodo col valore reale \(y\)
corrispondente al vettore di caratteristiche \(\mathbf{x}\): se i due
valori concidono, quindi la predizione è esatta, allora i pesi non
vengono aggiornati, sennò lo sono secondo una semplice formula che
prevede di sommare al valore della precedente esecuzione, il prodotto
tra un valore costante \(\pm 2\eta\), positivo se il valore corretto era
\(+1\), negativo in caso contrario, con \(\mathbf{x}\). Ogni esecuzione
è una \texttt{iterazione} di un ciclo che prende in ingresso un nuovo
campione e il valore dei pesi testé calcolato e continua fino a che non
diventi vera una \texttt{condizione\ di\ terminazione}.\\
Ma quando terminano le iterazioni? L'addestramento terminerà quando la
somma degli errori su tutti i campioni sarà nullo, oppure al disotto di
un valore prefissato. Nel mentre, i campioni dell'insieme di
addestramento saranno usati tutti in un gruppo di iterazioni, definito
come \texttt{epoca}, il cui numero è, evidentemente, pari al numero di
campioni dell'insieme di addestramento.\\
Al termine dell'addestramento sarà ottenuto un insieme di pesi, la cui
numerosità è pari a quella delle caratteristiche, e tali valori saranno
usati, senza alcuna ulteriore modifica, in tutte le esecuzioni della
seguente fase di generalizzazione (Figura 2), in cui useremo il
percettrone per ottenere classi a priori ignote. \{\{
insert\_image(target, `2', image\_location,
`Percettrone-Diagramma-generalizzazione.png', `Figura 2: Diagramma del
percettrone in fase di generalizzazione', `Diagramma del percettrone in
fase di generalizzazione') \}\}

\section{Funzione di attivazione}\label{funzione-di-attivazione}

Innanzitutto, definiamo formalmente la funzione di attivazione a gradino
\(\mathcal{H}\) (che è una versione modificata della
\texttt{funzione\ di\ Heaviside}): \[
\mathcal{H}(z)=\begin{cases} 1, & z≥θ \\ -1, & z<θ \end{cases}\tag*{\color{blue}{1}}
\] il cui grafico è mostrato in Figura 3. Il significato è il seguente:
se il valore dell'argomento della funzione è maggiore o uguale a
\(\theta\), allora essa assumerà il valore di \(+1\). Se invece
l'argomento sarà minore di \(\theta\), il risultato sarà \(-1\). \{\{
insert\_image(target, `3', image\_location,
`Percettrone-Funzione-attivazione-gradino.png', `Figura 3: Funzione di
attivazione a gradino', `Funzione di attivazione a gradino') \}\}

Nella definizione è presente una soglia predefinita \(\theta\), che
permette di inserire un grado di libertà molto importante
nell'addestramento, di cui approfondiremo nel seguito. L'argomento della
funzione di attivazione è \(z=\sum_{j=1}^{n} w_j x_j\), ottenuto come
somma pesata delle caratteristiche in ingresso alla iterazione, ove
\(n\) è il numero delle caratteristiche.\\
Generalmente, si preferisce una formulazione del tutto equivalente in
cui \(\mathcal{H}\) ha il gradino nello \(0\) dell'asse delle ascisse. A
tal fine, definiamo \(\bar{z}=z-θ\) e riformuliamo la funzione di
attivazione: \[
\mathcal{H}(\bar{z})=\begin{cases} 1, & \bar{z}≥0 \\ -1, & \bar{z}<0 \end{cases},\ \bar{z}=\sum_{j=1}^{N} w_j x_j-θ\tag*{\color{blue}{2}}
\] e definendo \(w_0=-θ\) (chiamato \textbf{\texttt{deriva}}) e
\(x_0=1\), quindi rinominando \(\bar{z}\) in \(z\), si ricava: \[
\mathcal{H}(z)=\begin{cases}1, & z≥0 \\ -1, & z<0 \end{cases},\ z=\sum_{j=0}^{N} w_j x_j\tag*{\color{blue}{3}}
\] dove, sostanzialmente, abbiamo esteso la sommatoria per comprendere
anche \(\theta\).\\
Questa definizione può essere resa in formato più compatto, introducendo
i vettori \(\mathbf{w}\) per i pesi e \(\mathbf{x}\) per le
caratteristiche: \[
\mathbf{w}=\begin{pmatrix} w_0 \\ w_1 \\ \vdots \\ w_n \end{pmatrix},\ \mathbf{x}=\begin{pmatrix} 1 \\ x_1 \\ \vdots \\ x_n \end{pmatrix}\tag*{\color{blue}{4}}
\] dove ricordiamo che le caratteristiche `reali' del modello sono
sempre in numero di \(n\), mentre i due vettori hanno dimensione
\(n+1\). Pertanto, la funzione di attivazione può essere riscritta
utilizzando il prodotto scalare tra i due vettori dei pesi e delle
caratteristiche, cioè
\(z=\sum_{j=0}^{N} w_j x_j=\langle\mathbf{w},\mathbf{x}\rangle\): \[
\mathcal{H}(z)=\begin{cases}1, & z≥0 \\ -1, & z<0 \end{cases},\ z=\langle\mathbf{w},\mathbf{x}\rangle.\tag*{\color{blue}{5}}
\]

\section{Regola di apprendimento}\label{regola-di-apprendimento}

Nella fase di addestramento, il percettrone `apprende' i pesi che
saranno usati in generalizzazione, cioè l'algoritmo determina il vettore
\(\mathbf{w}\), sotto la \texttt{supervisione} di un `insegnante' che
fornisce l'insieme di addestramento e osserva le predizioni (donde la
caratteristica dell'algoritmo di
\texttt{apprendimento\ supervisionato}).\\
Introduciamo alcune notazioni: l'insieme di addestramento \(\Xi\) sia
composto da \(N\) elementi \((\mathbf{x}^i, y_i)\) (i campioni), ognuno
con due valori, il vettore delle \(n+1\) caratteristiche reali
\(\mathbf{x}^i\) e la sua classe \(y^i\) che può assumere solo i valori
\(\pm1\): \[
(\mathbf{x}^1, y_1), \dots, (\mathbf{x}^N, y_N)\in\Xi\tag*{\color{blue}{6}}
\] con \(N\) numero positivo, \(\eta\) sia il
\textbf{\texttt{tasso\ di\ apprendimento}} con \(0<\eta≤1\), \(K\) sia
un numero intero positivo che identifichi l'iterazione corrente, e
\(\bar{y}_K\) denoti il risultato dell'algoritmo all'iterazione \(K\),
cioè la classe predetta, allora i passi che il percettrone esegue nella
fase di apprendimento, sono riassunti nella cosiddetta
\textbf{\texttt{regola\ di\ apprendimento}}:

\begin{quote}
\textbf{\texttt{Regola\ di\ apprendimento\ del\ percettrone}} 1.
Inizializza i pesi \(\mathbf{w}^0\) a \(0\) oppure a piccoli valori
casuali, definisci un tasso di apprendimento \(\eta\) e un numero
massimo di iterazioni \(\bar{K}\) o un numero massimo di epoche
\(\bar{E}\). 2. Per ogni epoca \(E\): 1. Per ogni campione
\((\mathbf{x}^\kappa, y_\kappa)\) dell'insieme di addestramento
(\(1≤\kappa≤N\) è l'indice positivo sull'insieme di addestramento, \(E\)
è l'indice positivo di epoche, \(N\) la dimensione dell'insieme di
addestramento e \(K=EN+\kappa\) il totale delle iterazioni compiute): 1.
Calcola la predizione della classe corrispondente a
\(\mathbf{x}^\kappa\): \[
     \bar{y}_K=\mathcal{H}\left(\sum_{j=0}^{n} w_j^{K-1} x_j^\kappa\right).\tag*{\color{blue}{7}}
     \] 2. Aggiorna i pesi \(\mathbf{w}^K\) secondo la formula: \[
      \mathbf{w}^K=\mathbf{w}^{K-1}+\eta(y_\kappa-\bar{y}_K)\mathbf{x}^\kappa.\tag* {\color{blue}{8}}
      \] 2. Continua fino alla prima occorrenza di una tra le due
condizioni seguenti: non vi siano più errori di predizione sull'insieme
di addestramento, cioè \(y_\kappa-\bar{y}_K=0\ \forall \kappa\), cioè in
una intera epoca, oppure finché non sia raggiunto il numero massimo di
iterazioni \(\bar{K}\) o il numero massimo di epoche \(\bar{E}\) (dove
\(\bar{K}=N\bar{E}\)). 3. Prendi il vettore dei pesi \(\mathbf{w}\) per
classificare caratteristiche con etichetta ignota nella fase di
generalizzazione.
\end{quote}

Il tasso di apprendimento \(\eta\) e il numero massimo di iterazioni
\(\bar{K}\) sono detti \textbf{\texttt{iperparametri}}, dato che non
sono modificati dalla regola di apprendimento, ma ricavati con tecniche
aggiuntive ad hoc. Le iterazioni sono tali da applicare tutto l'insieme
di addestramento un certo numero intero di volte, quindi
\(\bar{K}=E\cdot N\).\\
Si nota che: * Se il percettrone predice correttamente la classe, allora
\(y_\kappa-\bar{y}_K=0\), quindi i pesi rimangono invariati. * Se il
percettrone effettua una predizione erronea, allora \[
\mathbf{w}^K-\mathbf{w}^{K-1}=\eta (y_\kappa-\bar{y}_K)\mathbf{x}^\kappa=\begin{cases}\eta (1-(-1))\mathbf{x}^\kappa=2\eta \mathbf{x}^\kappa & y_\kappa=1,\ \bar{y}_K=-1 \\ \eta (-1-(1))\mathbf{x}^\kappa=-2\eta \mathbf{x}^\kappa & y_\kappa=-1,\ \bar{y}_K=1 \end{cases}\tag*{\color{blue}{9}}
\] quindi per \(y_\kappa=1\) i pesi si incrementano di una frazione
positiva del vettore delle caratteristiche, mentre per \(y_\kappa=-1\)
di una frazione negativa, la cui entità dipende da \(\eta\).

\section{Interpretazione geometrica}\label{interpretazione-geometrica}

Possiamo interpretare la \(\color{blue}\fbox{5}\), come l'esistenza di
una frontiera, per cui se un vettore delle caratteristiche è `sopra' di
essa, allora la classe corrispondente è \(+1\) , se al disotto allora
sarà \(-1\). Tale frontiera è definita da
\(\langle\mathbf{w},\mathbf{x}\rangle=0\) o, equivalentemente
\(\sum_{j=0}^{n} w_j x_j=0\), che possiamo riscrivere, ricordando che
\(x_0=1\), come \(w_0+x_1w_1+\dots+x_nw_n=0\) che corrisponde ad un
punto su una retta per \(n=1\), una retta nel piano per \(n=2\), un
piano nello spazio tridimensionale per \(n=3\) e un iperpiano in
\(\mathbb{R}^n\) per \(n>3\).\\
Scegliamo \(n=2\) per poter visualizzare i vettori delle caratteristiche
e la frontiera. Un vettore \(\mathbf{x}\), così come definito in
\(\color{blue}\fbox{4}\), lo disegniamo come un punto di coordinate
\((x_1,x_2)\) e la frontiera di equazione \(w_0+x_1w_1+x_2w_2=0\), come
una retta. Sappiamo che, ad ogni iterazione, i tre pesi \(w_0,w_1,w_2\)
possono assumere dei nuovi valori e, al termine dell'addestramento,
avranno quelli definitivi da usare nella fase di generalizzazione.\\
Nella Figura 4, tutti i punti sopra o sulla retta (sfondo rosa) sono
tali che \(w_0+x_1w_1+x_2w_2≥0\), quindi hanno classe \(+1\), mentre per
quelli per cui è \(w_0+x_1w_1+x_2w_2<0\) la classe è \(-1\) (sfondo
azzurro). \{\{ insert\_image(target, `4', image\_location,
`Percettrone-Frontiera-decisione.png', `Figura 4: Spazio bidimensionale
delle caratteristiche e frontiera di decisione', `Spazio bidimensionale
delle caratteristiche e frontiera di decisione') \}\}

Quindi, la retta \(w_0+x_1w_1+x_2w_2=0\) si comporta come una vera e
propria \textbf{\texttt{frontiera\ di\ decisione}}, giacché la posizione
relativa del vettore delle caratteristiche rispetto alla retta,
determina la classe predetta che gli corrisponde. Proprio perché la
frontiera di decisione è una linea, allora il percettrone è definito
come un \textbf{\texttt{classificatore\ lineare}}.\\
In pratica, l'addestramento consiste nel muovere la retta (o per un
numero di caratteristiche \(n>2\), un piano o un iperpiano) nello spazio
bidimensionale (rispettivamente, nello spazio tridimensionale o
multidimensionale), in modo da dividerlo perché tutti i punti con classe
\(+1\) siano sopra la frontiera e i rimanenti al disotto. Ciò non è
detto sia a priori possibile, a causa della distribuzione delle
caratteristiche, cioè non esista tale frontiera come nel caso della
Figura 5. \{\{ insert\_image(target, `5', image\_location,
`Percettrone-Caratteristiche-non-lin-sep.png', `Figura 5:
Caratteristiche non linearmente separabili', `Caratteristiche non
linearmente separabili') \}\}

Ciò si esprime dicendo che l'insieme di addestramento non è
\textbf{\texttt{linearmente\ separabile}} o, in altre parole, che il
percettrone commetterà sempre degli errori di classificazione in fase di
addestramento, giacché, per costruzione, può solo produrre frontiere di
decisione lineari. Al contrario, in Figura 4, è mostrato un insieme di
addestramento linearmente separabile con una possibile, tra le infinite,
frontiera di separazione che efficacemente distingue le caratteristiche
(in Figura 6 alcuni esempi di frontiere di decisione alternative). \{\{
insert\_image(target, `6', image\_location,
`Percettrone-Frontiere-multiple.png', `Figura 6: Frontiere di decisione
multiple', `Frontiere di decisione multiple') \}\}

\section{Deriva}\label{deriva}

Il termine deriva è usato in matematica e in informatica per
rappresentare una sorta di correzione, che viene aggiunta al risultato
di un algoritmo. Nella regola di apprendimento del percettrone, si nota
che senza la deriva \(w_0\), la frontiera di decisione passerebbe sempre
per l'origine dello spazio delle caratteristiche e ciò ne limiterebbe
evidentemente la capacità di separazione di sottoinsiemi degli insiemi
di apprendimento.\\
Per visualizzare questa affermazione, poniamo sempre \(n=2\) e facendo
riferimento alla Figura 6, dove è disegnata la frontiera
\(w_0+x_1w_1+x_2w_2=0\) con la sua intersezione con l'asse delle
ordinate \(-\frac{w_0}{w_2}\) che si può esprimere anche con
\(\frac{\theta}{w_2}\), ci si può convincere che per \(w_0=0\) o
\(\theta=0\), la frontiera passa per l'origine delle coordinate dello
spazio delle caratteristiche. \{\{ insert\_image(target, `7',
image\_location, `Percettrone-Frontiera-decisione-intercetta.png',
`Figura 7: Frontiera di decisione e deriva', `Frontiera di decisione e
deriva') \}\}

\section{Convergenza}\label{convergenza}

Sappiamo che una condizione necessaria perché il percettrone possa
classificare correttamente l'insieme di addestramento, sia che questo
abbia la proprietà di lineare separabilità. Quello che non sappiamo è se
ciò sia anche sufficiente, cioè se è garantito che la regola di
apprendimento produca un vettore \(\mathbf{w}\) senza errori di
classificazione e, quindi, sia costruita una frontiera di separazione
efficace. E questa è proprio la tesi del teorema di convergenza del
percettrone pubblicato per la prima volta da Rosenblatt nel 1958
({[}RF58II{]} e {[}RF62{]}). \textgreater{}
\textbf{\texttt{Teorema\ di\ convergenza\ del\ percettrone}}\\
\textgreater{} \textgreater{} Per ogni insieme di addestramento,
composto da un numero finito di campioni, l'algoritmo di addestramento
del percettrone produrrà un vettore \(\mathbf{w}\) che classificherà
correttamente tutti tali elementi, in un numero finito di epoche.

Ciò si esprime anche dicendo che l'algoritmo \textbf{\texttt{converge}}
e, usualmente, ciò accade in un numero di iterazioni che è di gran lunga
superiore alla dimensione dell'insieme di addestramento. Se la
condizione del teorema non è soddisfatta, allora l'algoritmo continuerà
a fare aggiustamenti ai pesi e alla deriva, senza mai raggiungere una
soluzione che classifichi correttamente tutti i campioni dell'insieme di
addestramento. In questo caso, si dice che l'algoritmo non converge.\\
Un secondo teorema, più tecnico, dimostrato indipendentemente da Henry
Block {[}BH62{]} e Aleksey Novikov {[}NA62{]}, stabilisce un limite
superiore al numero di errori di classificazione, il che ci permette di
stimare il tempo necessario all'algoritmo per convergere nel caso di
insieme di addestramento linearmente separabile.

\begin{quote}
\textbf{\texttt{Teorema\ del\ numero\ massimo\ di\ errori\ del\ percettrone\ (Block/Novikov)}}

Assumiamo che l'insieme di addestramento \(\Xi\) sia linearmente
separabile con margine \(\gamma\) e che la lunghezza (o norma euclidea)
di tutti i vettori delle caratteristiche, sia minore o uguale di un
certo valore \(R\) finito (o, più formalmente,
\(\lVert\mathbf{x}^i\rVert≤R\ \forall i\), per \(R>0\)). Quindi, il
massimo numero di errori compiuti dall'algorimo del percettrone in fase
di addestramento, è limitato superiormente da
\(\frac{R^2\lVert\mathbf{w}\rVert^2}{\gamma^2\rVert\widetilde{\mathbf{w}}\lVert^2}\),
con \(\mathbf{w}\) vettore dei pesi a convergenza e
\(\widetilde{\mathbf{w}}\) corrispondente vettore normale alla frontiera
di separazione.
\end{quote}

Nel teorema si cita il \textbf{\texttt{margine}} che è definito come la
distanza minima delle caratteristiche dalla frontiera di decisione in
\(\mathbb{R}^n\).\\
Quindi, adesso sappiamo che il numero massimo di epoche di addestramento
è limitato superiormente e se si commettesse almeno un errore per epoca
(se gli errori fossero zero allora l'algoritmo avrebbe raggiunto la
convergenza), allora il numero di epoche sarebbe inferiore a
\(\frac{R^2\lVert\mathbf{w}\rVert^2}{\gamma^2\rVert\widetilde{\mathbf{w}}\lVert^2}\).
Ciò è senz'altro confortante ma potrebbe portare ad un valore molto alto
e, d'altronde, l'utilità maggiore del teorema di Block-Novikov è nel
notare che il limite al numero massimo di errori che esso definisce, è
indipendente da: * Numero di dimensioni dello spazio delle
caratteristiche. * Dimensione dell'insieme di addestramento, cioè dal
numero dei campioni. * Tasso di apprendimento fintantoché sia mantenuto
costante (meno ovvio perché non citato nella tesi, ma ipotesi nella
dimostrazione). * Valore del vettore di inizializzazione di
\(\mathbf{w}\). * Ordine dei campioni dell'insieme di addestramento.

Al contrario, è dipendente da: * `Difficoltà' nella separazione, cioè
avere \(\gamma\) molto bassi (rispetto ai valori delle caratteristiche)
incrementa il limite. * La dispersione dei campioni (che porta ad un
\(R\) elevato).

Il teorema di Block-Novikov è interessante perché pone l'accento
sull'importanza di alcune proprietà dell'insieme di addestramento
rispetto ad altre, ma è, al contempo, poco utile operativamente.
Infatti, per stimare il limite superiore agli errori abbiamo bisogno di
fornire il margine \(\eta\), che non è noto a priori e non è un
risultato dell'addestramento!\\
D'altronde, anche ottenendo una stima della frontiera di decisione e del
margine, comunque non avremmo alcuna garanzia che siano `vicine' ai
valori di convergenza ottenuti dall'addestramento dell'algoritmo e,
soprattutto, né che tali valori siano ottimali, cioè tali da avere il
margine massimo o massimizzare una qualsiasi altra metrica di `qualità',
sia in addestramento che nella fase seguente di generalizzazione. In
generale, il percettrone convergerà ad una delle possibili frontiere di
decisione che separano i due insiemi di caratterstiche corripondenti a
classi diverse (come visualizzato in Figura 6).

\section{Generalizzazione}\label{generalizzazione}

Dopo l'addestramento supervisionato, il percettrone è pronto per
classificare ingressi di cui la classe è ignota. La fase diventa di
\textbf{\texttt{generalizzazione}}, definita come la capacità di
estendere la conoscenza del modello acquisita per mezzo dei campioni
dell'insieme di addestramento, a nuovi ingressi di cui le relative
classi sono sconosciute, coll'obiettivo di predire proprio quest'ultime,
in modo corretto. Questa finalità è la ragione per cui il percettrone e
tutti gli altri algoritmi di classificazione, sono sviluppati.\\
Possiamo definire una regola di generalizzazione, al pari di quella di
apprendimento, per definire il calcolo della predizione dato un nuovo
ingresso di caratteristiche.

\begin{quote}
\textbf{\texttt{Regola\ di\ generalizzazione\ del\ percettrone}} Dato un
vettore di caratteristiche \(\mathbf{x}\), la predizione sarà pari a: \[
\bar{y}=\mathcal{H}\left(\sum_{j=0}^{n}w_j x_j\right).\tag*{\color{blue}{10}}
\]
\end{quote}

D'altronde, la capacità del percettrone è limitata dall'essere un
classificatore lineare, il che significa che può solo apprendere
relazioni lineari tra le caratteristiche, quindi se la loro relazione
intrinseca fosse non lineare, esso non sarebbe comunque in grado di
apprenderla accuratamente e, quindi, avrebbe una scarsa capacità di
generalizzazione, o, in altre parole, commetterebbe un certo numero di
errori di predizione. Inoltre, anche la qualità dei dati di
addestramento ha un impatto sulla capacità di generalizzazione. Se i
campioni contengono errori o rumore, o se non sono rappresentativi
dell'intera popolazione, il percettrone non avrà una base informativa
sufficiente.\\
Un altro fattore che influisce sulla capacità di generalizzazione è la
complessità del modello fisico, che è legata al numero di
caratteristiche. Se il numero di caratteristiche è troppo grande
rispetto al numero di campioni, l'algoritmo può finire per
\texttt{sovradattarsi} ai dati di addestramento, il che significa che
apprende perfettamente i dati di addestramento, ma mostra una efficacia
limitata o, comunque, nettamente inferiore nella fase di validazione,
intermedia tra addestramento e generalizzazione vera e propria. In
generale, esistono diverse cause di sovradattamento, che sono meglio
connotate con algoritmi più complessi, ma quello che è importante
sottolineare è che il percettrone non ne è immune anche se,
concettualmente, è un algoritmo molto semplice.\\
L'errore commesso nella fase di validazione, è denominato
\textbf{\texttt{errore\ di\ generalizzazione}} e il suo calcolo si
effettua prendendo un insieme di campioni di caratteristiche e relative
classi, non usate in fase di addestramento, e valutando le predizioni
senza che i pesi e la deriva siano ulteriormente aggiornati. Quello che
si ottiene rappresenta una metrica di qualità dell'algoritmo nel suo
complesso, cioè, contemporaneamente, della scelta delle caratteristiche
del modello e della regola di apprendimento utilizzata, cioè
dell'algoritmo.

\section{Insiemi di addestramento non
separabili}\label{insiemi-di-addestramento-non-separabili}

Se l'insieme di addestramento non è linearmente separabile allora, per
sua definizione, non esiste alcuna frontiera di decisione lineare che
possa dividere l'insieme in due regioni dove, in ognuna, siano presenti
solo caratteristiche corrispondenti alla medesima classe. In questo
caso, il percettrone non potrà convergere e, ad ogni iterazione
nell'addestramento, produrrà una frontiera con un certo numero di errori
associati, qualsiasi sia il numero di iterazioni eseguite. Questa è una
forma di \textbf{\texttt{sottoadattamento}}, cioè il modello è troppo
semplice per catturare la struttura sottostante dei dati e ciò in un
modo aprioristico: è l'algoritmo utilizzato che non lo permette.\\
Già Marvin Minsky e Seymour Papert, in {[}MP69{]}, avevano mostrato che,
in caso di separabilità non lineare, il percettrone ricalcolerà
continuamente lo stesso vettore di pesi, quindi entrando in un ciclo
infinito. Ciò comporta la necessità di algoritmi più avanzati per
classificare correttamente tali insiemi di addestramento, come le
\textbf{\texttt{macchine\ a\ vettori\ di\ supporto}}, o altri metodi più
avanzati del percettrone.

\section{Ordine dei campioni in fase di
addestramento}\label{ordine-dei-campioni-in-fase-di-addestramento}

Durante l'addestramento, il percettrone aggiorna iterativamente i suoi
pesi in base agli errori commessi. Questi aggiornamenti sono guidati
dalla sequenza di campioni di addestramento forniti. Ecco perché
l'ordine in cui questi campioni vengono presentati, può influenzare
significativamente la velocità di convergenza e tecniche di modifica di
tale ordine riducono significativamente le iterazioni necessarie al
raggiungimento della soglia di arresto.\\
Rispetto alla regola di apprendimento presentata precedentemente, un
solo cambiamento viene effettuato e cioè l'aggiunta di un passo
nell'iterazione sulle epoche, in cui adottiamo una strategia di
modifica, o anche una combinazione, dell'ordine dei campioni, che può
includere anche una modifica nel numero di iterazioni nell'epoca, nel
caso in cui uno o più campioni vengano omessi o presentati più volte.

\begin{quote}
\textbf{\texttt{Regola\ di\ apprendimento\ del\ percettrone\ con\ ordine\ dei\ campioni\ modificato}}
1. Inizializza i pesi \(\mathbf{w}^0\) a \(0\) oppure a piccoli valori
casuali, definisci un tasso di apprendimento \(\eta\) e un numero
massimo di iterazioni \(\bar{K}\) o un numero massimo di epoche
\(\bar{E}\). 2. Per ogni epoca \(E\): 1. Modifica l'ordine dei campioni
dell'insieme di addestramento 2. Per ogni campione
\((\mathbf{x}^\kappa, y_\kappa)\) dell'insieme di addestramento
(\(1≤\kappa≤N\) è l'indice positivo sull'insieme di addestramento, \(E\)
è l'indice positivo di epoche, \(N\) la dimensione dell'insieme di
addestramento e \(K=EN+\kappa\) il totale delle iterazioni compiute): 1.
Calcola la predizione della classe corrispondente a
\(\mathbf{x}^\kappa\): \[
     \bar{y}_K=\mathcal{H}\left(\sum_{j=0}^{n} w_j^{K-1} x_j^\kappa\right).\tag*{\color{blue}{11}}
     \] 2. Aggiorna i pesi \(\mathbf{w}^K\) secondo la formula: \[
      \mathbf{w}^K=\mathbf{w}^{K-1}+\eta(y_\kappa-\bar{y}_K)\mathbf{x}^\kappa.\tag* {\color{blue}{12}}
      \] 3. Continua fino alla prima occorrenza di una tra le due
condizioni seguenti: non vi siano più errori di predizione sull'insieme
di addestramento, cioè \(y_\kappa-\bar{y}_K=0\ \forall \kappa\), cioè in
una intera epoca, oppure finché non sia raggiunto il numero massimo di
iterazioni \(\bar{K}\) o il numero massimo di epoche \(\bar{E}\) (dove
\(\bar{K}=N\bar{E}\)). 3. Prendi il vettore dei pesi \(\mathbf{w}\) per
classificare caratteristiche con etichetta ignota nella fase di
generalizzazione.
\end{quote}

Abbiamo aggiunto il passo 2.1 che a seconda delle strategie adottate
verrà esploso nelle istruzioni necessarie, anche complesse. Alcune
strategie: * Rimescolamento degli indici: gli indici vengono permutati
con un algoritmo come quello di Fisher-Yates che garantisce una
permutazione imparziale, cioè ogni possibile permutazione dell'elenco
degli indici ha la stessa probabilità di occorrenza. È implementato
dalla funzione \texttt{random.shuffle} in Python. Vi sono alternative
più complesse. * Ciclo: in pratica ad ogni epoca si parte dall'elemento
successivo a quello usato nella precedente (gli indici sono ottenuti
dalla operazione \(i+1 \bmod n\), se \(i\) è l'indice dell'ultima
sequenza). * Massima distanza: ad ogni epoca dividiamo in due
sottoinsiemi l'insieme di addestramento: quello coi campioni non
correttamente classificati e il complementare. Il primo lo ordiniamo per
distanza decrescente
\(\frac{-y_\kappa\langle\mathbf{w}^K, \mathbf{x}^\kappa\rangle}{\lVert\widetilde{\mathbf{w}}^K\rVert}\)
dalla ultima frontiera di decisione (quindi il primo sarà quello `più'
erroneamente classificato). Il secondo sottoinsieme per distanza
crescente
\(\frac{y_\kappa\langle\mathbf{w}^K, \mathbf{x}^\kappa\rangle}{\lVert\widetilde{\mathbf{w}}^K\rVert}\)
dalla frontiera di decisione o rimescolati. L'ordine nell'epoca sarà
dato dal primo sottoinsieme seguito dal secondo. * Massimo residuo: si
ordinano i campioni sulla base dell'errore, in valore assoluto, commesso
prima dell'applicazione della funzione di attivazione alla predizione
\(|\langle\mathbf{w}^K,\mathbf{x}^\kappa\rangle-y^\kappa|\), a partire
dal massimo e poi via via in modo decrescente.

Da notare come le prime due strategie siano indipendenti dalle
caratteristiche, dato che agiscono sui soli indici, mentre quelle di
massima distanza e residuo, dipendano dai campioni. Inoltre, possono
essere anche combinate come, ad esempio, la massima distanza per i
campioni classificati erroneamente e il rimescolamento per quelli
correttamente classificati.

\section{Oltre la classificazione
binaria}\label{oltre-la-classificazione-binaria}

Nel caso di problemi con un numero di classi maggiore di due, possiamo
usare delle tecniche di riduzione al caso binario, tra cui: *
\textbf{\texttt{Uno\ contro\ tutti}}: Addestriamo tanti percettroni
quante classi e ognuno sarà associato ad una classe che diventerà il
`caso positivo' (classe \(+1\)) e tutte le altre corrisponderanno al
`caso negativo' (classe \(-1\)). Tutti gli insiemi di addestramento
risulteranno \textbf{\texttt{sbilanciati}}, se l'insieme di partenza
avrà un numero di campioni per classe simile. Ciò, in generale, vedremo
essere una condizione non ottimale. *
\textbf{\texttt{Uno\ contro\ uno}}: per ogni coppia di classi,
addestriamo un percettrone e, pertanto, in totale saranno
\(\binom{\Sigma}{2}=\frac{\Sigma(\Sigma-1)}{2}\), ove \(\Sigma\) è il
numero delle classi. Ciò significa che il numero di percettroni aumenta
col quadrato del numero di classi, ma l'addestramento di ognuno è su un
numero di campioni inferiore.

Identificheremo le varie alternative di riduzione al caso binario sotto
il termine comune di \textbf{\texttt{percettrone\ multiclasse}}.

\subsection{Uno contro tutti}\label{uno-contro-tutti}

\subsubsection{Regola di apprendimento}\label{regola-di-apprendimento-1}

L' insieme di dati di addestramento ha sempre la forma
\(\color{blue}\fbox{6}\) con l'unica differenza che adesso \(y_i\) può
assumere \(\Sigma\) valori anziché due, con \(\Sigma>2\). Chiameremo il
percettrone che classifica la classe \(\sigma\)-esima (con
\(1≤\sigma≤\Sigma\)) come `percettrone \(\sigma\)' e i suoi parametri e
iperparametri avranno \(\sigma\) come apice o pedice.\\
Definiamo le \(\Sigma\) classi come \(\alpha_\sigma\) e l'insieme
contenente tali classi come \(A\), quindi la funzione di attivazione del
percettrone \(\sigma\) diventa: \[
\bar{\mathcal{h}}=\mathcal{H}(\langle\mathbf{w}_\sigma, \mathbf{x}\rangle) = 
\begin{cases} 
1 & \implies \bar{y}=\alpha_\sigma, & \langle\mathbf{w}_\sigma,\mathbf{x}\rangle \geq 0 \vphantom{\underset{i\neq\sigma}{\lor}},\ &\alpha_{\sigma}\in A \\
-1 & \implies \bar{y}=\smash{\underset{i\neq\sigma}{\lor}}\bar{\alpha}_i, &\langle\mathbf{w}_\sigma, \mathbf{x}\rangle < \theta
\end{cases}
\tag*{\color{blue}{13}}
\] cioè se l'argomento della funzione è non negativo, allora la classe
predetta \(\bar{y}\) è \(\alpha_\sigma\), mentre se l'argomento è
negativo, allora la predizione ci dice solo che è una delle classi
diverse da \(\alpha_\sigma\), ma non quale tra le classi appartenenti
all'insieme \(A\setminus\{\alpha_{\sigma}\}\) di tutti gli elementi di
\(A\) diversi da \(\alpha_{\sigma}\).\\
Prima di continuare con la regola di apprendimento, facciamo un esempio
con tre etichette: `rosso', `verde' e `blu' a cui corrisponderanno le
classi \(\alpha_{rosso}\), \(\alpha_{verde}\) e \(\alpha_{blu}\).
Quindi, avremo tre percettroni e ad esempio per il `rosso':\\
\[
\bar{\mathcal{h}}=\mathcal{H}_{rosso}(\langle\mathbf{w}_{rosso}, \mathbf{x}\rangle)=\begin{cases} 1\ &\implies \bar{y}=\alpha_{rosso}, & \langle\mathbf{w}_{rosso}, \mathbf{x}\rangle≥0 \\ -1\ &\implies \bar{y}\ =\alpha_{verde}\lor\alpha_{blu}, & \langle\mathbf{w}_{rosso}, \mathbf{x}\rangle<θ \end{cases}\tag*{\color{blue}{14}}
\] da cui si evince che il `percettrone rosso' predice o la classe con
etichetta rosso o quella con etichetta verde o blu. Per semplicità
abbiamo sostituito alla classe l'etichetta, considerata la
corrispondenza biunivoca tra di esse. Pertanto, la regola di
apprendimento diviene:

\begin{quote}
\textbf{\texttt{Regola\ di\ apprendimento\ del\ percettrone\ multiclasse\ \textquotesingle{}uno\ contro\ tutti\textquotesingle{}}}
1. Inizializza i pesi \(\mathbf{w}^0_\sigma\) per ogni percettrone
\(\sigma\), con \(\sigma \in \Sigma\), a \(0\) oppure a piccoli valori
casuali, definisci un tasso di apprendimento \(\eta\) e un numero
massimo di epoche \(\bar{E}\) o una soglia minima di errore come somma
di tutti gli errori commessi dai percettroni o definita dal massimo tra
tutti i percettroni. 2. Per ogni epoca \(E\): 1. Per ogni campione
\((\mathbf{x}^\kappa, y_\kappa)\) dell'insieme di addestramento
(\(\kappa\) con \(1≤\kappa≤N\) è l'indice positivo sull'insieme di
addestramento \(\Xi\) che ha dimensione \(N\), \(E\) è l'indice positivo
di epoche e \(K=EN+\kappa\), il totale delle iterazioni compiute): 1.
Per ogni percettrone \(\sigma\): 1. Calcola la predizione corrispondente
all'ingresso \(\mathbf{x}^\kappa\): \[
          \bar{\mathcal{h}}_K=\mathcal{H}\left(\langle\mathbf{w}^{K-1}_{\sigma},\mathbf{x}^\kappa\rangle\right)\tag*{\color{blue}{15}}
          \] 2. Ricava dalla classe `vera' \(y_{\kappa}\in A\) il
corrispondente valore \(\mathcal{h}_{\kappa}=\pm1\), usando la
corrispondenza tra \(1\) e \(\alpha_{\sigma}\) e \(,-1\) e
\({\underset{i\neq\sigma}{\lor}}\bar{\alpha}_i\): \[
          \begin{align}y_{\kappa}=\alpha_{\sigma}&\implies h_{\kappa}=1\\y_{\kappa}\neq\alpha_{\sigma}&\implies h_{\kappa}=-1\end{align}.\tag*{\color{blue}{16}}
          \] 3. Aggiorna i pesi \[
          \mathbf{w}^K_{\sigma}=\mathbf{w}^{K-1}_{\sigma}+\eta(\mathcal{h}_{\kappa}-\bar{\mathcal{h}}_K)\mathbf{x}^{\kappa}\tag*{\color{blue}{17}}.
          \] 2. Ripeti il passo 1 per un numero predefinito di epoche o
fino a quando l'errore ad una certa iterazione non abbia raggiunto la
soglia predefinita. 3. Prendi i \(\Sigma\) vettori di
pesi~\(\mathbf{w}_\sigma\)~per classificare caratteristiche con
etichetta ignota nella fase di generalizzazione.
\end{quote}

Ritorniamo all'esempio dell'insieme colle etichette rosso, verde e blu.
Supponiamo che le corrispondenti classi siano 1, 2, 3. Supponiamo che
\(y_{\kappa}=2\) (etichetta verde), allora la regola di addestramento
all'iterazione \(K\) si traduce in: * Percettrone rosso: se
\(\langle\mathbf{w}^{K-1}_{rosso},\mathbf{x}^\kappa\rangle≥0\), allora
predice \(+1\), quindi rosso, cioè commette un errore. D'altronde il
valore vero dell'etichetta è verde quindi \(h_K=-1\) e i pesi sono
effettivamente aggiornati
\(\mathbf{w}^K_{rosso}=\mathbf{w}^{K-1}_{rosso}-2\eta\mathbf{x}^{\kappa}\).
Se \(\langle\mathbf{w}^{K-1}_{rosso},\mathbf{x}^\kappa\rangle<0\) allora
i pesi rimangono invariati, giacché la predizione è corretta. *
Percettrone verde: se
\(\langle\mathbf{w}^{K-1}_{verde},\mathbf{x}^\kappa\rangle≥0\), allora
predice \(+1\), quindi verde, correttamente e i pesi non sono
aggiornati. Se
\(\langle\mathbf{w}^{K-1}_{verde},\mathbf{x}^\kappa\rangle<0\) allora
predice rosso o blu, quindi commette un errore. I pesi sono da
aggiornare con \(h_K=1\) cioè
\(\mathbf{w}^K_{verde}=\mathbf{w}^{K-1}_{verde}+2\eta\mathbf{x}^{\kappa}\).
* Percettrone blu: come il rosso.

\subsection{Generalizzazione}\label{generalizzazione-1}

Supponiamo che il percettrone multiclasse abbia terminato addestramento,
adesso può essere impiegato su caratteristiche non già `viste'. Prima
però dobbiamo definire una sorta di regola di generalizzazione che, dato
un ingresso, permetta di ottenere la classe di uscita.\\
Se tutti i percettroni hanno raggiunto la convergenza e supponendo che
la classe corrispondente al nuovo ingresso \(\mathbf{x}\) sia
\(\alpha_{\hat{\sigma}}\), allora il percettrone \(\hat{\sigma}\)
dovrebbe classificarlo con \(+1\) e tutti gli altri con \(-1\): \[
\begin{align}\langle\mathbf{w}_{\sigma},\mathbf{x}\rangle&≥0, \ \sigma=\hat{\sigma}\\\langle\mathbf{w}_{\sigma},\mathbf{x}\rangle&<0,\ \forall\sigma\neq\hat{\sigma} \end{align}\tag*{\color{blue}{18}}
\] che, in modo più compatto, si può esprimere con \[
\hat{\sigma}=\underset{1≤\sigma≤\Sigma}{\mathrm{argmax}}\left(\langle\mathbf{w}_{\sigma},\mathbf{x}\rangle\right).\tag*{\color{blue}{19}}
\] Tornando all'esempio delle etichette dei colori, supponiamo di avere
un campione \((x,2)\) (dove \(2\) era la classe dell'etichetta verde),
allora, sempre nella condizione di classificazione corretta da parte di
tutti i percettroni: * Percettrone rosso:
\(\langle\mathbf{w}_{rosso},\mathbf{x}\rangle<0\). * Percettrone verde:
\(\langle\mathbf{w}_{verde},\mathbf{x}\rangle≥0\). * Percettrone blu:
\(\langle\mathbf{w}_{blu},\mathbf{x}\rangle<0\).

La \(\color{blue}\fbox{19}\) in questo caso restituisce \(2\), cioè
l'etichetta verde, dato che il massimo è tra un valore positivo (quello
del percettrone verde) e due negativi, quindi il comportamento della
formula è coerente con le nostre aspettative.\\
In generale, la \(\color{blue}\fbox{19}\) è la formula utilizzata per la
generalizzazione del percettrone multiclasse in versione `uno contro
tutti', per determinare la classe corrispondente ad un nuovo ingresso e,
come abbiamo visto, nel caso ideale si comporta correttamente.\\
D'altronde, se uno o più percettroni non hanno raggiunto la convergenza
o l'insieme di addestramento non è linearmente separabile, oppure, nella
condizione ideale, comunque commettono un errore di predizione, allora
non varranno più le \(\color{blue}\fbox{18}\) nel senso che sia il
percettrone corrispopndente alla classe `vera' potrebbe fallire, o gli
altri classificarlo erroneamente. In tutti questi casi, la
\(\color{blue}\fbox{19}\) è la scelta più ragionevole perché sfrutta il
massimo della conoscenza acquisita dal percettrone multiclasse.\\
Va però notato che non tutte le regioni dello spazio delle
caratteristiche portano alla definizione di una unica classe come
risultato.

\subsection{Uno contro uno}\label{uno-contro-uno}

\subsubsection{Regola di apprendimento}\label{regola-di-apprendimento-2}

In questo caso addestreremo un percettrone per ogni coppia di classi
distinte \(\upsilon\) e \(\tau\) tra le \(\Sigma\), e lo chiameremo
percettrone \(\upsilon\tau\).

\begin{quote}
\textbf{\texttt{Regola\ di\ apprendimento\ del\ percettrone\ multiclasse\ \textquotesingle{}uno\ contro\ uno\textquotesingle{}}}
1. Per ogni percettrone \(\upsilon\tau\), seleziona dall'insieme di
addestramento \(\Xi\) solo i campioni che hanno \(\upsilon\) e \(\tau\)
come classi e tale sottoinsieme sia \(\Xi_{\upsilon\tau}\) e il campione
generico \((\mathbf{x}_{\upsilon\tau},y_{\upsilon\tau})\). 2.
Inizializza i pesi \(\mathbf{w}^0_{\upsilon\tau}\), per ogni percettrone
\(\upsilon\tau\), a \(0\) oppure a piccoli valori casuali, definisci un
tasso di apprendimento \(\eta\) e un numero massimo di epoche
\(\bar{E}\) o una soglia minima di errore come somma di tutti gli errori
commessi dai percettroni o definita dal massimo tra tutti i percettroni.
3. Per ogni epoca \(E\): 1. Per ogni percettrone \(\upsilon\tau\): 1.
Per ogni campione
\((\mathbf{x}^\kappa_{\upsilon\tau}, y_{\upsilon\tau,\kappa})\)
dell'insieme di addestramento (\(\kappa\) con
\(1≤\kappa≤N_{\upsilon\tau}\) è l'indice positivo sull'insieme di
addestramento \(\Xi_{\upsilon\tau}\) , \(E\) è l'indice positivo di
epoche, \(N_{\upsilon\tau}\) la dimensione dell'insieme di addestramento
e \(K=EN_{\upsilon\tau}+\kappa\) il totale delle iterazioni compiute):
2. Calcola la predizione della classe corrispondente a
\(\mathbf{x}^\kappa_{\upsilon\tau}\): \[
          \bar{y}_{\upsilon\tau,K}=\mathcal{H}\left(\sum_{j=0}^{n} w_{\upsilon\tau,j}^{K-1} x_{\upsilon\tau,j}^\kappa\right).\tag*{\color{blue}{20}}
          \] 3. Aggiorna i pesi \(\mathbf{w}^K_{\upsilon\tau}\) secondo
la formula: \[
          \mathbf{w}^K_{\upsilon\tau}=\mathbf{w}^{K-1}_{\upsilon\tau}+\eta(y_{\upsilon\tau,\kappa}-\bar{y}_{\upsilon\tau,K})\mathbf{x}^\kappa_{\upsilon\tau}.\tag* {\color{blue}{21}}
          \] 2. Ripeti il passo 1 per un numero predefinito di epoche o
fino a quando l'errore ad una certa iterazione non abbia raggiunto la
soglia predefinita. 4. Prendi i \(\binom{\Sigma}{2}\) vettori di
pesi~\(\mathbf{w}_{\upsilon\tau}\)~per classificare caratteristiche con
etichetta ignota nella fase di generalizzazione.
\end{quote}

\subsubsection{Generalizzazione}\label{generalizzazione-2}

La formula di generalizzazione conta i `voti' dei percettroni binari e
la classe con più voti, è quella risultante della classificazione. In
particolare, per ogni percettrone \(\upsilon\tau\), si calcola la
predizione \(\bar{y}_{\upsilon\tau}\) e si assegna un voto alla classe
che corrisponde al risultato (quindi 1 voto a \(\upsilon\) o \(\tau\)),
e così si ottiene elenco di classi con relativo totale di voti di cui
prendere il massimo.\\
Per quanto sia semplice, tale procedura porta a delle ambiguità, perché
nel caso due o più classi ottengano lo stesso numero di voti, sarà
necessario un ulteriore criterio per ridurre tale molteplicità ad una.

\subsection{Utilità}\label{utilituxe0}

Il percettrone multiclasse è stato presentato al fine di mostrare alcune
possibili estensioni del binario, ma le limitazioni di quest'ultimo
appaiono come amplificate e tali da rendere la versione multiclasse poco
interessante, rispetto ad alternative come le macchine a vettori di
supporto, gli alberi di decisione o anche le reti neurali profonde, per
citare solo alcuni tra gli strumenti di classificazione più potenti.\\
Inoltre, gli algoritmi di apprendimento o di predizione non aggiungono
granché su un piano strettamente didattico, per questo rimandiamo ad una
trattazione più completa nelle prossime lezioni.

\section{Elementi chiave della
lezione}\label{elementi-chiave-della-lezione}

\begin{enumerate}
\def\labelenumi{\arabic{enumi}.}
\tightlist
\item
  Abbiamo introdotto alcuni concetti importanti dell'apprendimento delle
  macchine e cioè i classificatori lineari, l'apprendimento
  supervisionato, la regola e l'insieme di apprendimento, la
  generalizzazione e il suo errore, la deriva, la funzione di
  attivazione. Essendo concetti generali è importante che siano compresi
  profondamente, sfruttando la semplicità dell'algoritmo del
  percettrone.
\item
  Altri concetti, ugualmente importanti come la frontiera di decisione e
  il margine appartenenti al novero dei classificatori, saranno
  ugualmente ripresi, anche se con un corredo di nozioni più articolato.
\item
  Abbiamo formulato l'algoritmo del percettrone in un modo che possa
  essere implementato in un linguaggio di programmazione arbitario.
\end{enumerate}

\section{Prossimo passo}\label{prossimo-passo}

Percettrone - Implementazione Python Percettrone - Temi Avanzati
Macchine a vettori di supporto

\section{Per approfondire}\label{per-approfondire}

\subsection{Documenti storici sul
percettrone}\label{documenti-storici-sul-percettrone}

Il progetto PARA fu lanciato da Rosenblatt al Cornell Aeronautical
Laboratory nel 1956 col duplice obiettivo: dimostrare la fattibilità di
una nuova tecnica statistica e testare la `funzionalità' di quella
tecnica su hardware ad hoc, come il MARK I. In particolare, mirava a
stabilire la fattibilità tecnica ed economica di un `analogo del
cervello', in grado di riconoscere strutture simili di informazioni
ottiche, elettriche o sonore. Negli anni 1956-1962 approfondirà in
diversi report, pubblicazioni e libri il contesto teorico del
percettrone, generando anche significative aspettative.\\
Il percettrone, così come presentato da Rosenblatt e da altri autori, è
piuttosto diverso da quello formalizzato sopra, perché le
generalizzazioni teoriche seguenti hanno `depurato' le supposte analogie
con il cervello umano, con un formalismo che permette la focalizzazione
sulle proprietà matematiche di algoritmi simulabili su macchine
enormemente più potenti rispetto a quelle degli anni '50 e '60. Inoltre,
Rosenblatt intendeva il percettrone come una macchina che completò nel
1960 e chiamò Mark I, anche se usò l'IBM 704 per effettuare una
simulazione informatica e mostrare le potenzialità dell'algoritmo
({[}CM21{]}).

{[}RF58I{]} Rosenblatt, Frank
(1957).~\texttt{The\ Perceptron\ —\ A\ Perceiving\ and\ Recognizing\ Automaton\ (Project\ PARA)}.
Tech. Rep.~85-460-1. Cornell Aeronautical Laboratory. \emph{La prima
pubblicazione con il termine percettrone e gli obiettivi di ricerca di
Rosenblatt.} \textgreater{} \emph{Recent theoretical studies by this
writer indicate that it should be feasible to construct an electronic or
electromechanical system which will learn to recognize similarities or
identities between patterns of optical, electrical, or tonal
information, in a manner which may be closely analogous to the
perceptual processes of a biological brain. The proposed system depends
on probabilistic rather than deterministic principles for its operation,
and gains its reliability from the properties of statistical
measurements obtained from large populations of elements. A system which
operates according to these principles will be called
a~\textbf{perceptron}. (Pag. 2)} \textgreater{} \textgreater{}
\emph{Recenti studi teorici di questo autore indicano che dovrebbe
essere fattibile costruire un sistema elettronico o elettromeccanico che
imparerà a riconoscere somiglianze o identità tra modelli di
informazioni ottiche, elettriche o sonore, in un modo che può essere
strettamente analogo ai processi percettivi di un cervello biologico. Il
sistema proposto si basa su principi probabilistici piuttosto che
deterministici per il suo funzionamento, e guadagna la sua affidabilità
dalle proprietà delle misure statistiche ottenute da grandi popolazioni
di elementi. Un sistema che opera secondo questi principi sarà chiamato
perceptron. (Pag. 2)}

{[}RF58I{]} Rosenblatt, Frank
(1958).~\texttt{The\ perceptron:\ A\ theory\ of\ statistical\ separability\ in\ cognitive\ systems\ (Project\ PARA)}.
U.S. Dept. of Commerce, Office of Technical Services.

{[}RF58II{]} Rosenblatt, Frank (1958).
\texttt{The\ perceptron:\ A\ probabilistic\ model\ for\ information\ storage\ and\ organization\ in\ the\ brain}.
Psychological Review.~65~(6):
386--408.~DOI:\href{https://doi.org/10.1037\%2Fh0042519}{10.1037/h0042519}.

{[}RF58III{]} Rosenblatt, Frank (1958).
\texttt{Two\ theorems\ of\ statistical\ separability\ in\ the\ perceptron\ (Project\ PARA)}.~Cornell
Aeronautical Laboratory, Inc.

{[}RF62{]} Rosenblatt, Frank
(1962).~\texttt{Principles\ of\ neurodynamics:\ perceptrons\ and\ the\ theory\ of\ brain\ mechanisms}.~Spartan
Books.

{[}BH62{]} Block, Henry (1962).
\texttt{The\ Perceptron:\ A\ Model\ for\ Brain\ Functioning.\ I}.
Reviews of Modern Physics. 34, 123-135. DOI:
\href{https://doi.org/10.1103/RevModPhys.34.123}{10.1103/RevModPhys.34.123}.

{[}NA62{]} Novikov, Albert B. J. (1962).
\texttt{On\ convergence\ proofs\ on\ perceptrons}. Proceedings of the
Symposium on the Mathematical Theory of Automata. Vol. XII: 615--622.

{[}MP69{]} Minsky, Marvin \& Papert, Seymour (1969).
\texttt{Perceptrons:\ An\ Introduction\ to\ Computational\ Geometry}.
MIT Press. ISBN: 0-262-63022-2/978-0-262-63022-1.

{[}MP17{]} Minsky, Marvin \& Papert, Seymour (2017).
\texttt{Perceptrons:\ An\ Introduction\ to\ Computational\ Geometry.\ Reissue\ of\ the\ 1988\ Expanded\ Edition\ with\ a\ new\ foreword\ by\ Léon\ Bottou}.
MIT Press. ISBN: 0-262-53477-0/978-0-262-53477-2.

\subsection{Storia dell'apprendimento delle
macchine}\label{storia-dellapprendimento-delle-macchine}

{[}NN09{]} Nilsson, Nils J.
(2009).~\texttt{The\ Quest\ for\ Artificial\ Intelligence}. Cambridge
University Press. ISBN: 9780511819346. DOI:
\href{https://doi.org/10.1017/CBO9780511819346}{10.1017/CBO9780511819346}.
\href{https://ai.stanford.edu/~nilsson/QAI/qai.pdf}{PDF}

{[}CM21{]} Metz, Cade (2021).
\texttt{Genius\ Makers:~The\ Mavericks\ who\ Brought\ AI\ to\ Google,\ Facebook\ and\ the\ World}.
Dutton. ISBN: 9781847942135. \emph{Ed. It}. (2022).
\texttt{Costruire\ l\textquotesingle{}intelligenza.\ Google,\ Facebook,\ Musk\ e\ la\ sfida\ del\ futuro}.
Mondadori. ISBN: 9788804745839.

\subsection{Manuali}\label{manuali}

{[}AS23{]} Axler, Sheldon (2023). \texttt{Linear\ Algebra\ Done\ Right}.
Springer. DOI:
\href{https://doi.org/10.1007/978-3-031-41026-0}{10.1007/978-3-031-41026-0}.
ISBN: 978-3-031-41025-3.
\href{https://link.springer.com/content/pdf/10.1007/978-3-031-41026-0.pdf}{PDF}

{[}MR22{]} Hardt, Moritz \& Recht, Benjamin (2022).
\texttt{Patterns,\ Predictions,\ and\ Actions:\ Foundations\ of\ Machine\ Learning}.
Princeton University Press. ISBN: 9780691233734.
\href{https://mlstory.org/pdf/patterns.pdf}{PDF}

\bookmarksetup{startatroot}

\chapter{Summary}\label{summary}

In summary, this book has no content whatsoever.

\bookmarksetup{startatroot}

\chapter*{References}\label{references}
\addcontentsline{toc}{chapter}{References}

\markboth{References}{References}

\phantomsection\label{refs}
\begin{CSLReferences}{0}{1}
\end{CSLReferences}



\end{document}
